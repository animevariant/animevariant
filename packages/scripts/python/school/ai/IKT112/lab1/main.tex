\documentclass{article}
\usepackage{amsmath}
\begin{document}

\title{Exercise 1: Signal Processing}
\maketitle

\begin{enumerate}
    \item \textbf{What is the relationship between the frequency and the period of a periodic signal?}
    The relationship is defined by:
    \[ f = \frac{1}{T} \]
    where \( T \) is the period. Frequency is defined by how many periods pass in a unit of time, for example, 3 periods per second or 48 periods per second. Therefore, the relationship between them is that frequency is inversely proportional to the period.

    \item \textbf{Express \( \sin(\omega t) \) using cosine.}
    \[ \sin(\omega t) = \cos\left(\omega t - \frac{\pi}{2}\right) \]

    \item \textbf{What is the magnitude of the signal \( 1.8 \sin(300\pi t) \)?}
    Magnitude: 1.8

    \item \textbf{What is the frequency of the signal \( 0.5 \cos(400\pi t) \)?}
    Frequency: 200 Hz

    \item \textbf{A system has a signal input and a signal output. How can we see if the system is linear or non-linear by putting a sine signal onto the system?}
    If you send a sinus signal into a system and receive an output with a different frequency, then the system is non-linear. If you send the same sine signal and receive the same frequency in the output, then the system is linear. In both cases, the amplitude and the phase can change.
\end{enumerate}

\end{document}
