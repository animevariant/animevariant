\documentclass{article}
\usepackage{amsmath}
\usepackage{amssymb}

\begin{document}

\title{Analysis of Clock Photographs, Signal Sampling, and Discrete Sequences}
\author{Student}
\date{\today}
\maketitle

\section*{Clock Photographs Analysis}

\begin{enumerate}
    \item Taking photographs every 55 minutes of a clock with only a minute hand will make it appear that time is moving backwards. The minute hand will be 5 minutes behind its position in the previous photo, as in 55 minutes, the minute hand does not complete a full hour (60 minutes) cycle.
    \item To show proper clockwise rotation of the minute hand, photos should be taken every 12 minutes or less. Therefore, taking at least 4 or more photos per hour is correct, as 60 minutes divided by 4 gives 15 minutes, which is within the 12-minute requirement.
\end{enumerate}

\section*{Analysis of Sampled Signal \( x(t) \)}

\begin{enumerate}
    \item The important missing parameter for analyzing the sampled signal \( x(t) \) is the sampling rate. Without knowing the sampling rate, the analysis of the signal in the time or frequency domain is limited.
\end{enumerate}

\section*{Discrete Sequence Equations}

\begin{enumerate}
    \item For a sampling frequency of 4002Hz, the correct equation for \( x(n) \) is \( x(n) = \cos\left(\frac{\pi}{2} \cdot n + \frac{\pi}{7}\right) \).
    \item For a sampling frequency of 1502Hz, the correct equation for \( x(n) \) is \( x(n) = \cos\left(\frac{4\pi}{3} \cdot n + \frac{\pi}{7}\right) \).
\end{enumerate}

\section*{Signal \( x(t) \) Analysis}

\textbf{Assuming the signal is \( x(t) = \cos(5 \pi \cdot 400t) \):}

\begin{enumerate}
    \item The number of samples for a full rotation depends on the actual frequency of the signal. With a frequency of 2000Hz, the number of samples is not 5 for a full rotation.
    \item The phase change per sample depends on the frequency of the signal and the sampling rate. The calculation should be based on the actual frequency of the signal.
\end{enumerate}

\end{document}
